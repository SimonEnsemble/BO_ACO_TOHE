\documentclass[11pt, oneside]{article}  
% \documentclass[fleqn,10pt]{wlpeerj}


\usepackage{geometry}  
%\usepackage{nunito}
\usepackage{cmbright}
\geometry{letterpaper}   
\usepackage{cite}
\usepackage{graphicx}				% Use pdf, png, jpg, or eps§ with pdflatex; use eps in DVI mode
								% TeX will automatically convert eps --> pdf in pdflatex	
\usepackage{tcolorbox}
\usepackage{amssymb}
\usepackage{longtable}
\usepackage{amsmath}
\usepackage{booktabs}
% \usepackage{emoji}
\usepackage{url}
\usepackage{color}
\definecolor{c1}{rgb}{0.12, 0.56, 1.0}
%SetFonts
\usepackage{xspace}
\usepackage{xcolor}
\usepackage{caption}
\usepackage{subcaption}
\usepackage{authblk} 

\usepackage{sectsty} 
\definecolor{cool_blue}{RGB}{24, 132, 193}
\sectionfont{\color{c1}\selectfont}
\subsectionfont{\color{c1}\selectfont}
\subsubsectionfont{\color{c1}\selectfont}
% \paragraphfont{\color{gray}\selectfont}
\subparagraphfont{\color{gray}\selectfont}

\definecolor{fruitpushorange}{RGB}{255, 127, 0}
\newcommand{\data}{$(s_1, ..., s_k)$\xspace}

\usepackage{soul}
\DeclareRobustCommand{\cms}[1]{ {\begingroup\sethlcolor{fruitpushorange}\hl{(cms:) #1}\endgroup} }
%SetFonts

\sloppy 

\title{Bi-objective ant colony optimization of the risky, robot-team orienteering problem}
\author[1]{Cory M. Simon}
\author[2]{Jeffrey Richley}
\author[2]{Lucas Overbey}
\author[2]{Darleen Perez-Lavin}
\affil[1]{School of Chemical, Biological, and Environmental Engineering. Oregon State University. Corvallis, OR. USA.}
% \affil[]{\texttt{cory.simon@oregonstate.edu}}
\affil[2]{Naval Information Warfare Center Atlantic. Charleston, SC. USA.}
% \corrauthor[1]{Cory M. Simon}{cory.simon@oregonstate.edu}

% \keywords{Keyword1, Keyword2, Keyword3}



%\flushbottom
% \maketitle
%\thispagestyle{empty}


%\affil[*]{}
% \date{}							% Activate to display a given date or no date

\begin{document}
\maketitle

\begin{abstract}
In many applications e.g., resource delivery, patrolling, and information-gathering, a team of mobile [aerial, ground, or aquatic] robots must coordinate their trails in some environment to cooperatively achieve some team-level objective. Robots may risk failure/destruction while traversing some environments, owing to dangerous conditions or the presence of adversaries capable of attacking them. Then, robot trail-planning should account for these risks.

Herein, we use ant colony optimization to find the [approximate] Pareto-optimal set of robot trail plans for the bi-objective, risky team orienteering problem, where (i) a team of robots are mobile within an environment abstracted as a graph (nodes: locations, edges: paths between locations); (ii) each node offers a reward to the team when visited by a robot; (iii) the traversal of each edge imposes a risk of robot failure/destruction; and (iv) the two [often, competing] team objectives are to maximize the expected (a) team reward and (b) number of robots that survive the mission. 
Presenting the Pareto-optimal set of robot trail plans to the downstream decision-maker allows them to choose the plans that balance, according to their values, the two objectives. As a case study, we illustrate with an information-gathering mission in a nuclear power plant from a Defense Advanced Research Projects Agency (DARPA) robots challenge.
\end{abstract}

\clearpage

\section{Introduction}
Mobile [aerial, ground, or aquatic] robots equipped with sensors, actuators, and/or cargo have applications in agriculture (eg. planting and harvesting crops, spraying pesticide, monitoring crop health) \cite{santos2020path}, commerce (eg.\ order fulfillment in warehouses) \cite{wurman2008coordinating}, the delivery of goods \cite{coelho2014thirty}, search-and-rescue \cite{queralta2020collaborative}, chemical, biological, radiological, or nuclear incident response (eg.\ safely localizing the source(s) and assessing the severity of the incident) \cite{murphy2012projected}, environmental monitoring \cite{dunbabin2012robots}, safety monitoring of an industrial chemical plant  \cite{soldan2014towards}, forest fire monitoring and fighting \cite{merino2012unmanned}, and military surveillance and reconnaissance. 
Deploying a \emph{team} of robots, as opposed to a single robot, can increase spatial coverage, decrease the time to achieve the mission, and give resilience to the failure of a robot.
Often, we wish for the team of mobile robots to coordinate their paths in the environment to cooperatively achieve a shared, team-level objective \cite{parker2007distributed,lesser1999cooperative}.


In some applications, the team of mobile robots must traverse an environment that poses risks of robot failures, due to inherently dangerous terrain, harsh weather, the presence of heat, radiation, or corrosive chemicals, mines, piracy, or an adversary with the capability to attack/disable/destroy robots. Then, the robots should coordinate their paths in consideration of these risks, so that achievement of the team objective is resilient to robot failures \cite{zhou2021multi}. 
A \emph{resilient} team of robots \cite{prorok2021beyond} can handle robot failures by 
(i) adopting a policy that anticipates failures and thus can withstand failures with minimal concession of the objective
or
(ii) adapting/reorganizing in response to realized failures of robots on the team to recoup loss in the objective. 
%  (i) withstand failures/attacks with minimal concession of the objective and/or (ii) adapt to realized failures/attacks of robots on the team to maximize the objective. 

Several models and algorithms have been developed for multiple robots to coordinatively plan their paths in risky environments abstracted as graphs (nodes: locations; edges: paths between locations) \cite{jorgensen2018team,shi2023robust,zhou2022distributed}. 
For example, in the Team Surviving Orienteers (TSO) problem \cite{jorgensen2018team}, each node of the graph offers a reward to the team when visited by a robot, but each edge traversal by a robot poses a risk of failure. The objective in the [offline] TOP is to plan the paths of the robots (from a source node to a destination node) to maximize the expected team reward under the constraint that each robot survives the mission with a probability above a certain threshold. 
In the extended Matroid TSO problem \cite{jorgensen2017matroid,jorgensen2024matroid}, we seek to maximize the weighted
expected number of nodes visited by one or more robots.
Relatedly, the Foraging Route with the Maximum Expected Utility problem \cite{di2022foraging} considers a single robot foraging in an adversarial environment like in the TOP, but the rewards collected by the robot are not accumulated until the robot returns to the source node safely.
In the [offline] Robust Multiple-path Orienteering Problem \cite{shi2023robust}, nodes offer rewards if visited by a robot that survived the mission; the paths of the $N$ robots are planned to maximize the team reward under the worst-case attacks of $\alpha<N$ of the robots by an adversary. The optimal path plans must trade off (i) redundancy in the nodes visited to endow robustness against attacks and (ii) coverage of many nodes to collect many rewards.
Tangentially related work involves handling adversarial attacks on the sensors of the robots \cite{liu2021distributed} and maximizing coverage of an area with threats to robots \cite{korngut2023multi,yehoshua2016robotic}.

\section{The bi-objective risky team orienteering problem (BO-RTOP)}
In the risky team orienteering problem (RTOP), our task is to coordinatively plan the trails of a team of mobile robots on a directed graph whose (i) nodes offer rewards that are irrevocably harvested [for the team] by visiting robots and (ii) edges, when traversed by a robot, impose a risk of robot failure/destruction.
We consider the offline setting, where the trails are set at the beginning of the mission, then followed by the robots without updates during the mission to eg.\ adapt to robot failures. 
For the bi-objective RTOP (BO-RTOP), we wish to find the set of Pareto-optimal trail plans for the robot team that maximize the expected (i) number of robots that survive the mission and (ii) rewards collected by the team.

The BO-RTOP follows the TSOP setup in Ref.~\cite{jorgensen2018team} but with three changes: we (i) omit the constraints that each robot survives above a threshold probability, (ii) allow robots to follow trails instead of paths (the latter prohibits a robot from visiting a node more than once), and (iii) aim to maximize two objectives instead of one.

\subsection{Spatially abstracting the environment as a directed graph}
We abstract the environment as a directed graph $G=(\mathcal{V}, \mathcal{E})$. Each node $v\in \mathcal{V}$ represents a location. Each edge $(v, v^\prime) \in \mathcal{E}$ represents the best (e.g., shortest or safest) path (in Euclidean space) for a mobile robot to take from the location represented by node $v$ to the location represented by $v^\prime$.
Note, we do not assume the graph is complete\footnote{Ie., not every pair of distinct nodes $\{v, v^\prime\}$ is joined by two edges $(v, v^\prime)$ and $(v^\prime, v)$. 
Eg., consider a building with node $v$ representing a room on the first floor, node $v^\prime$ a room on the second floor, and node $u$ as the staircase between the first and second floors. Traveling from node $v$ to node $v^\prime$ necessitates passing through node $u$ first.}.

\subsection{A team of mobile robots}
A homogenous team of $K$ robots all begin at the base node $v_b \in \mathcal{V}$ of the graph $G$. The robots are \emph{mobile}, generally meaning they may walk on the graph $G$.

\subsection{The robot-team trail plan}
Each robot on the team plans to execute/follow a closed, directed trail $\rho$ on the graph $G$.  
The set of closed, directed trails $\{\rho_1, ..., \rho_K\}$ the robot-team plans to follow are the \emph{robot-team trail plan} for the mission---``plans'' because robots may fail/get destroyed while following these trails. 
Of course, any robot that fails before its trail is completed cannot visit the remainder of the nodes in its planned trail.

\subparagraph{A closed, directed trail.} A directed trail is a sequence of nodes $\rho=(\rho(0), \rho(1), ..., \rho(\lvert \rho \rvert))$ (so, $\rho(i) \in \mathcal{V}$) such that 
(i) for each node in the sequence except the last one, an edge is directed from that node to the next node in the sequence, ie., $(\rho(i-1), \rho(i))\in\mathcal{E}$ for $i \in \{1, ..., \lvert \rho \rvert \}$,
and 
(ii) the edges traversed in the trail are unique, ie. each edge in the multiset $\{(\rho(i-1), \rho(i))\}_{i=1}^{\ell}$ has a multiplicity of one.
That the trail is closed means the trail begins and ends at the same node, ie. $\rho(0)=\rho(\lvert \rho \rvert)$, which, here, $=v_b$.
The nodes in the trail need not be distinct.

\subparagraph{The static/offline setting.} 
The robot-team trail plan is set at the beginning of the mission, then followed by the robots without adaptation or updates during the mission eg. in reaction to realized robot failures. 
This corresponds to a situation where the robots cannot communicate with each other, nor can the command center send communications to the robots. 

\subsection{The model of the risk of robot failure/destruction during trail-following} 
Each robot incurs a risk of failure/destruction while following its path $\rho$. 
Specifically, starting (in a surviving state) at some node $v$, a robot survives the lumped process of (i) traversing edge $(v, v^\prime) \in \mathcal{E}$ and (ii) visiting node $v^\prime$ with probability $\omega(v, v^\prime)$. 
We assume (i) each outcome (success or failure) of an edge traversal followed by a node visit by a robot is an independent event and (ii) survival probabilities are static over the course of the mission. 
Thus, the robot survival probability function $\omega: \mathcal{E} \rightarrow [0, 1]$ characterizes the survival probabilities of the $K$ robots following trail plan $\{\rho_1, ..., \rho_K\}$ on the graph $G$.% and (2) the expected utility of the rewards harvested by the robots along their paths, which we write next. 

\subparagraph{Remark on symmetry.} We do not assume $\omega$ is symmetric, ie., that $\omega(v, v^\prime) = \omega(v^\prime, v)$. The traversal from node $v$ to $v^\prime$ may be more dangerous than the traversal from $v^\prime$ to $v$ owing to eg., (i) strong air or water currents in the direction $v^\prime$ to $v$ or (ii) a robot detector at node $v$ that can alert the adversary to attempt to intercept the robot during its outgoing journey. Even if edge traversal risks are symmetric, the action of visiting a node be risky, and node $v$ may be more or less dangerous than node $v^\prime$, breaking symmetry. 

\subparagraph{Interpretation of the robot survival probability function.} We model the survival/failure of a robot traversing an edge as stochastic owing to eg. (i) the unpredictable nature of eg. an aerial robot crashing into an obstacle or a ground robot getting stuck in rocks or mud or (ii) an adversary, present on the edge, with imperfect capability to (a) detect and/or (b) attack robots.

\paragraph{Survival probabilities.} 
We now derive some useful robot survival probabilities based on the risky-traversal model encoded in the robot survival probability function $\omega$.

\subparagraph{Survival of a robot following its trail.}
Let $S_n(\rho) : \{\text{fail}, \text{survive}\} \rightarrow \{0, 1\} $ be the Bernoulli random variable that is $1$ if a robot following trail $\rho$ survives to visit the $n$th node in this trail, and $0$ if it does not. For the event of survival, the robot must survive its traversal of \emph{all} of the first $n$ edges in its trail, so:
\begin{equation}
	\pi(S_n(\rho) = 1) = \prod_{i=1}^n \omega(\rho(i-1), \rho(i)) \text{ for } n\in \{1, ..., \lvert \rho \rvert\} \label{eq:pi_S_n}
\end{equation} The factorization owes to the independence of the edge-traversal$+$node-visit events.
The complement of the event of survival is failure, so $\pi(S_n(\rho) = 0)=1-\pi(S_n(\rho) = 1)$.

Of course, the outcome $S_n(\rho)=0$ implies $S_{n+i}(\rho)=0$ for $i \in \{1, ..., \lvert \rho \rvert - n\}$ since, once a robot fails along its trail, it cannot survive to visit nodes later in the trail.

\subparagraph{Survival of the team of robots following their trails.}
Let the random variable $R(\{\rho_1, ..., \rho_K\})$ be the number of robots that survive the mission, where the robot team follows trail plan $\{\rho_1, ..., \rho_K\}$. Owing to the independence of the edge-traversal events and thus of robot survival events, $R$ is the sum of the Bernoulli random variables giving the survival outcomes of each robot over its entire path:
\begin{equation}
	R(\{\rho_1, ..., \rho_K\})=\sum_{k=1}^K S_{\lvert \rho_k \rvert}(\rho_k). \label{eq:R_sum}
\end{equation}
Thus, $R$ follows the Poisson-Binomial distribution \cite{tang2023poisson}.
The probability that $r$ robots survive the mission is:
\begin{multline}
	\pi(R=r) = \sum_{\substack{\mathcal{R} \subseteq \{1, ..., K\}  \\ \lvert \mathcal{R} \rvert = r} } \,
	\prod_{k \in \mathcal{R}} \pi(S_{\lvert \rho_k \rvert}(\rho_k) = 1)
	\prod_{k \in \{1, ..., K\} \setminus \mathcal{R}} [1- \pi(S_{\lvert \rho_k \rvert}(\rho_k) = 1)], \\ \text{ for } r \in \{0, ..., K\}
	\label{eq:R_pb}
\end{multline} with $ \pi(S_{\lvert \rho_k \rvert}(\rho_k) = 1)$ in eqn.~\ref{eq:pi_S_n}.
Eqn.~\ref{eq:R_pb} sums over all $\binom{K}{r}$ possible size-$r$ surviving subsets $\mathcal{R}$ of the $K$ robots; the first product is the probability that all of those robots in $\mathcal{R}$ indeed survive their closed trails, and the second product is the probability that the other robots outside of $\mathcal{R}$ indeed fail somewhere along their closed trails.
Seen from eqn.~\ref{eq:R_sum}, the expected number of robots that survive the mission is:
\begin{equation}
	\mathbb{E}[R(\{\rho_1, ..., \rho_K\})]=\sum_{k=1}^K \mathbb{E}[S_{\lvert \rho_k \rvert}(\rho_k)] = \sum_{k=1}^K  \pi(S_{\lvert \rho_k \rvert}(\rho_k) = 1).
\end{equation}

\subparagraph{Visitation of a node by a robot following its trail.} 
Let the random variable $T_v(\rho)$ be the number of times a robot following trail $\rho$ visits node $v\in \mathcal{V} \setminus \{v_b\}$. 
% If node $v$ does not belong to the planned trail, of course, $T_v=0$. 
Equivalently, $T_v(\rho)$ is the number of edge-traversal events along the planned trail $\rho$ that (i) result in survival and (ii) land on node $v$ as the sink node. 
Thus, in terms of the robot-survival random variables $S_n(\rho)$:
\begin{equation}
	T_v(\rho) = \sum_{n \in \theta_v(\rho) } S_n(\rho), % \mathcal{I}[\rho(n) = v].
\end{equation}
where $\theta_v(\rho)$ is the strictly ordered set (list) of indices of the edges in the trail $\rho$ with sink node $v$:
\begin{equation}
	\theta_v(\rho) = (\{ n \in \{1, ..., \lvert \rho \rvert\} : \rho(n) = v\}, <).
\end{equation} So, $\lvert \theta_v(\rho) \rvert$ is the number of times node $v$ is planned to be visited by the robot following trail $\rho$, and successful traversal of the $\theta_v(\rho)(n)$th edge in the trail by the robot would be the $n$th time node $v$ is visited by that robot.
If node $v$ doesn't belong to the planned trail $\rho$ of the robot, the number of times node $v$ will be visited by the robot is zero with certainty ($T_v(\rho)=0$).
If node $v$ does belong to the trail $\rho$, the event that node $v$ is visited exactly $t$ times ($T_v(\rho)=t$) is equivalent to
(i) if $t=0$ (node $v$ is not visited), the event that the robot does not survive its first visit to node $v$, ie. $S_{\theta_v(\rho)(1)}=0$;
(ii) if $t=\lvert \theta_v(\rho)\rvert$ (node $v$ visitations equal to that in the plan), the event that the robot survives its last planned visit to node $v$, ie. $S_{\theta_v(\rho)(\lvert \theta_v(\rho) \rvert)}=1$;
(iii) if $1 \leq t < \lvert \theta_v(\rho ) \rvert$, the event that (a) the robot survives its $t$th planned visit to node $v$ and (b) does not survive its $(t+1)$th planned visit, ie. $(S_{\theta_v(\rho)(t)}=1) \land (S_{\theta_v(\rho)(t+1)}=0)$. So, the probability mass function of $T_v(\rho)$ is:
\begin{equation}
	\pi(T_v(\rho) = t ) = 
	\begin{cases}
		\begin{cases}
			1, & t = 0 \\
			0, & 1 \leq t \leq \lvert \theta_v(\rho ) \rvert  \\
		\end{cases}
		&  v \notin \rho \\
		\begin{cases}
			\pi(S_{\theta_v(\rho)(1)}=0), & t = 0 \\
			\pi\left( (S_{\theta_v(\rho)(t)}=1) \land (S_{\theta_v(\rho)(t+1)}=0)\right), &1 \leq t < \lvert \theta_v(\rho ) \rvert   \\
			\pi\left( (S_{\theta_v(\rho)(\lvert \theta_v(\rho) \rvert)}=1) \right), & t = \lvert \theta_v(\rho ) \rvert   \\
			0 , & t >  \lvert \theta_v(\rho ) \rvert
		\end{cases}
		& v \in \rho
	\end{cases}
\end{equation}
with
\begin{align}
\pi\left( (S_{\theta_v(\rho)(t)}=1) \land (S_{\theta_v(\rho)(t+1)}=0)\right) &=
\pi\left( S_{\theta_v(\rho)(t+1)}=0 \mid S_{\theta_v(\rho)(t)}=1)\right)
\pi (S_{\theta_v(\rho)(t)}=1) \\
&= \left(1-\prod_{i=\theta(t)+1}^{\theta(t+1)} \omega(\rho(i-1), \rho(i)) \right) \prod_{i=1}^{\theta(t)} \omega(\rho(i-1), \rho(i)).
\end{align}

\subparagraph{Visitation of a node by the team of robots following their trails.} 
The number of times node $v$ is visited by \emph{any} robot belonging to the robot team following trail plan $\{\rho_1, ..., \rho_K\}$ is
\begin{equation}
	T_v(\{\rho_1, ..., \rho_K\} ) = \sum_{k=1}^K T_v(\rho_k),
\end{equation} owing to independence of the robot survival events. 
Its probability mass function sums over all possible ways to split the $t$ visits of node $v$ among the $K$ robots:
\begin{equation}
	\pi(T_v(\{\rho_1, ..., \rho_K\} = t) = 
	% \sum_{t_1 \in \{0, ..., \lvert \theta_v(\rho_1) \rvert\} } \cdots  \sum_{t_K \in \{0, ..., \lvert \theta_v(\rho_K) \rvert\} } 
	\sum_{t_1 + \cdots t_K = t}
	\prod_{k=1}^K \pi(T_v(\rho_k)=t_k),
\end{equation} with $t_k \in \{ 0, ..., \lvert \theta_v(\rho_k) \rvert\} $ the number of times robot $k$ visits node $v$.

% The number of robots on the team, with trail plans $\{\rho_1,...,\rho_K\}$, that visit node $v$ is $\sum_{k=1}^K Z_v(\rho_k)$.
% HUGE MESS: if trails not paths, then can visit a node more than once. think of a star graph. then this is not a sum of independent variables.


\subsection{Total utility gained by the robot-team from node-visits}
Finally, the reason the robots are following their trails on the graph is to collect rewards from the nodes.
Each node $v\in \mathcal{V}$ in the graph $G$ offers rewards to the team if visited by robot(s) on the team. 
Let $u_v: \mathbb{N}_{\geq 0} \rightarrow \mathbb{R}_{\geq 0}$ be a utility function that maps 
the number of visits node $v$ receives by a robot over the course of the mission
 to 
 the additive reward accumulated by the team from that node.
Note, even if the robot fails after leaving a node, it counts as a visit to that node.
In other words, the reward offered by a node is immediately and irrevocably accumulated by the team after a visit by a robot on the team.
Then, the total utility collected by the robot-team following trail plan $\{\rho_1, ..., \rho_K\}$ is the random variable:
\begin{equation}
U(\{\rho_1,...,\rho_K\}) = \sum_{v\in\mathcal{V}\setminus \{v_b\}} u_v\left ( T_v(\{\rho_1, ..., \rho_K\}) \right).
\end{equation}
(The base node $v_b$ does not offer reward to the team.)

The expectation utility accumulated over the mission by a robot-team following trail plan $\{\rho_1, ..., \rho_K\}$ is:
\begin{equation}
	\mathbb{E}[U(\{\rho_1,...,\rho_K\})]= \sum_{v\in\mathcal{V}\setminus \{v_b\}} \sum_{t= 0}^{\sum_{k=1}^K \lvert \theta_v(\rho_k) \rvert } t u_v(t) \pi(T_v(\{\rho_1, ..., \rho_K\}) = t)
\end{equation}
% I can handle non-diminishing returns! e.g. if two visits better than one.

\subparagraph{Single-visit rewards.}
In the case of single-visit rewards, a node can offer reward to only a single robot. I.e., once a node is visited, it does not offer further rewards to the team. Then, the utility function is
\begin{equation}
	u_v(k) = \begin{cases}
		0 & k = 0 \\
		\upsilon_v & k \geq 1,
	\end{cases}
\end{equation} where $\upsilon_v \in \mathbb{R}_{\geq 0}$ is the reward offered by node $v$. And, the expected team-reward is:
\begin{align}
	\mathbb{E}[U(\{\rho_1, ..., \rho_K\}) & = \sum_{v \in \mathcal{V} \setminus \{v_b\}} \upsilon_v \pi(T_v(\{\rho_1, .., \rho_K\}) \geq 1) \\
		      & = \sum_{v \in \mathcal{V} \setminus \{v_b\}} \upsilon_v \left(1 - \prod_{k=1}^K  \pi(T_v(\rho_k) =0) \right)
\end{align} since the event that node $v$ is visited by one or more robots is the complement of the event that none of the robots on the team visit it.

\subparagraph{Multi-visit rewards.}

\subsection{The bi-objective optimization problem}
We wish to find the team-robot trail plan $\mathcal{P}:=\{\rho_1, ..., \rho_K\}$ that maximizes two objectives, (1) the expected team-reward and (2) the expected number of robots that survive the mission:
\begin{equation}
\max_{\mathcal{P}=\{\rho_1, ..., \rho_K\}} [\mathbb{E}[U(\mathcal{P})], \mathbb{E}[R(\mathcal{P})]].
\label{eq:the_two_objs}
\end{equation}

There may not exist a utopian robot-team trail plan that simultaneously maximizes both objectives. 
That is, these two objectives may compete, implying the ultimate team trail plan selected for the mission may depend on a tradeoff between the expected utility and the expected survival; making this tradeoff requires invoking values held by a (human) decision-maker. Intuitively, valuing survival more than utility will favor trail-plans where robots do not enter dangerous regions of the environment, even if large rewards are contained there, and valuing utility more than survival will instead send robots into these dangerous regions to attempt collection of those rewards. 

Owing to the inherent tradeoff between survivability and collected reward, herein we seek the Pareto-optimal set of team-robot trail plans. Then, we present the Pareto set to the decision-maker, who ultimately selects the team-robot trail-plans according to his or her values in terms of the importance of team-reward vs. robot survival. 

\paragraph{Pareto-optimal team-robot trail plans.} 
A Pareto-optimal robot-team trail plan $\mathcal{P}:\{\rho_1, ..., \rho_K\}$ cannot be altered to
(1) increase the survivability objective $E[R(\mathcal{P})]$ without compromising (decreasing) the utility objective $E[U(\mathcal{P})]$
nor
(2) increase the utility objective $E[U(\mathcal{P})]$ without compromising (decreasing) the survivability objective $E[R(\mathcal{P})]$.
Formally, a team trail plan $\mathcal{P}:=\{\rho_1, ..., \rho_K\}$ Pareto-dominates team trail plan  $\mathcal{P}^\prime :=\{\rho_1^\prime, ..., \rho_K^\prime\}$ if both
\begin{align}
	\mathbb{E}[U(\mathcal{P})] \geq \mathbb{E}[U(\mathcal{P}^\prime)] & \text{ and }  \mathbb{E}[R(\mathcal{P})] \geq \mathbb{E}[R(\mathcal{P}^\prime)] \\
	\mathbb{E}[U(\mathcal{P})] > \mathbb{E}[U(\mathcal{P}^\prime)] & \; \text{ or   } \; \mathbb{E}[R(\mathcal{P})] > \mathbb{E}[R(\mathcal{P}^\prime)].
\end{align}
A plan $\mathcal{P}$ belongs to the Pareto-optimal set of plans if no other trail plan $\mathcal{P}^\prime$ Pareto-dominates it.


% harvest risk different from visit risk. once harvested, then no longer risk for other robots to visit that node.

\section{Bi-objective ant colony optimization}
To find the [approximate] set of Pareto-optimal robot-team trail plans, with respect to the two objectives in eqn.~\ref{eq:the_two_objs}, we resort to bi-objective (BO) ant colony optimization (ACO) \cite{iredi2001bi}. 
ACO \cite{dorigo2006ant}, a form of swarm intelligence \cite{bonabeau1999swarm} inspired by ants foraging for food via pheromone trails \cite{bonabeau2000inspiration}, is a meta-heuristic for combinatorial optimization problems---particularly naturally-suited for finding optimal paths on graphs. 

% stigmergy
We have a colony of $N_{\text{ants}}$ artificial ants that search for Pareto-optimal team-robot trail plans. 
ACO proceeds in iterations. 
At each iteration, each ant in the colony constructs a robot-team trail plan $\{\rho_1, ..., \rho_K\}$. 
The ants collaborate in the search, in that they maintain a shared, global set of Pareto-optimal team-robot trail plans.

The ant colony is heterogeneous; each ant is assigned a parameter $\lambda \in [0, 1]$ that dictates how it will balance the two objectives as it searches for a Pareto-optimal team-robot trail plan. Particularly, ant $i\in\{1, ..., N_{\text{ants}}\}$ in the colony is assigned $\lambda_i := (i-1) / (N_{\text{ants}}-1)$. A $\lambda$ closer to zero (one) implies the ant prioritizes maximizing the expected utility $E[U]$ (expected survivability $E[R]$). The idea of having a heterogeneous colony is that each ant focuses on finding robot-team trail-plans belonging to different regions of the Pareto front. 

Each ant constructs a team-robot trail plan $\{\rho_1, ..., \rho_K\}$ sequentially, robot-by-robot; i.e., the ant fully constructs the closed trail for robot 1, $\rho_1$, then $\rho_2$, and so on. And, the construction of a trail by an ant proceeds node-by-node, starting at the base node $v_b$; i.e., the ant follows the trail the robot plans to take. Suppose the ant is constructing the closed trail for robot $k$, $\rho_k$, and currently sits at node $x=\rho_k(i)$, meaning that the ant has chosen the first $i$ nodes of robot $k$'s trail, with the partial trail being $\rho_k^\dagger=(v_b, \rho_k(1), ..., \rho_k(i))$. 
The next node $y=\rho_k(i+1)$ in the trail is chosen probabilistically among those viable, with probability defined according to (i) the amounts $\tau_u(x, y)$ and $\tau_r(x, y)$ of two species of pheromone on the edge from $x$ to $y$ that pertain to the ants' past experience of traversing this edge with respect to the $\mathbb{E}[U]$ and $\mathbb{E}[R]$ objective, (ii) our heuristics $\eta_u(x, y)$ and $\eta_r(x, y)$  that define the [greedy] appeal of taking the edge $u$ to $v$ with goals of optimizing the $\mathbb{E}[U]$ and $\mathbb{E}[R]$ objective, (iii) the previously constructed robot trails $\{\rho_1, ..., \rho_{k-1}\}$, and (iv) how the ant balances the two objectives, encoded in its $\lambda$:
\begin{equation}
	\pi(y \mid x) = 
	\begin{cases}
		\dfrac{
		 \left[\tau_r(x, y) \eta_r(x, y) \right]^\lambda \left[ \tau_u(x, y) \eta_u(x, y; \{\rho_1, ..., \rho_{k-1}\}) \right]^{1-\lambda} }{
		 \sum_{y^\prime \in \mathcal{V}^\dagger} \left[\tau_r(x, y^\prime) \eta_r(x, y^\prime) \right]^\lambda \left[ \tau_u(x, y^\prime) \eta_u(x, y^\prime; \{\rho_1, ..., \rho_{k-1}\}) \right]^{1-\lambda} 
		 }
		 &
		 y \in \mathcal{V}^\dagger
		  \\
		 0, & y \notin \mathcal{V}^\dagger
	\end{cases} \label{eq:prob_x_y}
\end{equation} where the set of viable next-nodes in the trail, $y=\rho(i+1)$, are those serving as a sink node for an edge (i) whose source node is $x=\rho(i)$ and (ii) has not already been taken in the partial trail $\rho_k^\dagger$:
\begin{equation}
 	\mathcal{V}^\dagger := \{ y : (x, y) \in \mathcal{E} \land \nexists j \in \{1, ..., i\} : (\rho_k^\dagger(j), \rho_k^\dagger(j+1))= (x, y)  \}.
\end{equation}
We next explain the pheromone and the heuristic.

\paragraph{Pheromone.} The ants lay two distinct species of artificial pheromone on the edges of the graph $G$---one species for each objective. Let the functions $\tau_r:\mathcal{E}\rightarrow \mathbb{R}_+$ ($\tau_u:\mathcal{E}\rightarrow \mathbb{R}_+$) give the amount of the pheromone species on the edges of the graph $G$ that represent their promise, learned from the past experience of ants, for belonging to robot trails that maximize the expected number of robots that survive $\mathbb{E}[R]$ (the expected team utility gained by the robots $\mathbb{E}[U]$). Thus, when selecting the next node for a trail, eqn.~\ref{eq:prob_x_y} specifies that ants are more likely to traverse an edge if it has a large amount of pheromone on it; depending on the ant's value of $\lambda$, its decision places more or less emphasis on $\tau_u$ vs. $\tau_r$. Note, the pheromone functions are not static, but change from iteration-to-iteration. 


\paragraph{Heuristic.}

\begin{align}
\eta_r(x, y) & := \omega(x, y) \\
\eta_u(x, y, \{\rho_1, ..., \rho_{k-1}\}) & :=  \omega(x, y) \mathbb{E}[u_v(T_v(\{\rho_1, ..., \rho_{k-1}\})+1)]
\end{align}
marginal expected reward


\paragraph{Laying pheromone.}

\section{Results}

\paragraph{DARPA power plant}

Defense Advanced Research Projects Agency (DARPA) Subterranean (SubT) Challenge \cite{chung2023into}


\section{Discussion}

Much future work remains.
The risky orienteering problem can be extended to account for more real-world complexity, such as (a) fuel/battery constraints for the robots and perhaps nodes that serve as fuel/recharging stations, (b) stochastic rewards

other risk metrics \cite{majumdar2020should}.

Future work. online


\bibliography{refs}
\bibliographystyle{unsrt}

\end{document}  
