\documentclass[11pt, oneside]{article}  
% \documentclass[fleqn,10pt]{wlpeerj}


\usepackage{geometry}  
%\usepackage{nunito}
\usepackage{cmbright}
\geometry{letterpaper}   
%\usepackage{cite}
\usepackage{graphicx}				% Use pdf, png, jpg, or eps§ with pdflatex; use eps in DVI mode
								% TeX will automatically convert eps --> pdf in pdflatex	
\usepackage{tcolorbox}
\usepackage{amssymb}
\usepackage{longtable}
\usepackage{amsmath}
\usepackage{booktabs}
% \usepackage{emoji}
\usepackage{url}
\usepackage{color}
\definecolor{c1}{rgb}{0.12, 0.56, 1.0}
%SetFonts
\usepackage{xspace}
\usepackage{xcolor}
\usepackage{caption}
\usepackage{subcaption}
\usepackage{authblk} 

\usepackage{sectsty} 
\definecolor{cool_blue}{RGB}{24, 132, 193}
\sectionfont{\color{c1}\selectfont}
\subsectionfont{\color{c1}\selectfont}
\subsubsectionfont{\color{c1}\selectfont}
% \paragraphfont{\color{gray}\selectfont}
\subparagraphfont{\color{gray}\selectfont}

\definecolor{fruitpushorange}{RGB}{255, 127, 0}
\newcommand{\data}{$(s_1, ..., s_k)$\xspace}

\usepackage{soul}
\DeclareRobustCommand{\cms}[1]{ {\begingroup\sethlcolor{fruitpushorange}\hl{(cms:) #1}\endgroup} }
%SetFonts


\title{Bi-objective ant colony optimization of the risky, robot-team orienteering problem}
\author[1]{Cory M. Simon}
\author[2]{Jeffrey Richley}
\author[2]{Lucas Overbey}
\author[2]{Darleen Perez-Lavin}
\affil[1]{School of Chemical, Biological, and Environmental Engineering. Oregon State University. Corvallis, OR. USA.}
% \affil[]{\texttt{cory.simon@oregonstate.edu}}
\affil[2]{Naval Information Warfare Center Atlantic. Charleston, SC. USA.}
% \corrauthor[1]{Cory M. Simon}{cory.simon@oregonstate.edu}

% \keywords{Keyword1, Keyword2, Keyword3}



%\flushbottom
% \maketitle
%\thispagestyle{empty}


%\affil[*]{}
% \date{}							% Activate to display a given date or no date

\begin{document}
\maketitle

\begin{abstract}
In many applications e.g., delivery, patrolling, and information-gathering, a team of mobile [aerial, ground, or aquatic] robots must coordinate their trails in some environment to cooperatively achieve some team-level objective. 
Robots may risk failure/destruction while traversing some environments, owing to dangerous conditions or the presence of adversaries capable of attacking them. 
Then, robot trail-planning should account for these risks.

Herein, we use ant colony optimization to find the [approximate] Pareto-optimal set of robot trail plans for the bi-objective, risky team orienteering problem, where (i) a team of robots are mobile within an environment abstracted as a graph (nodes: locations, edges: paths between locations); (ii) each node offers a reward to the team when visited by a robot; (iii) the traversal of each edge imposes a risk of robot failure/destruction; and (iv) the two [often, competing] team objectives are to maximize the expected (a) team reward and (b) number of robots that survive the mission.
Presenting the Pareto-optimal set of robot trail plans to the downstream decision-maker allows them to choose the plans that balance, according to their values, the two objectives. 
As a case study, we illustrate with an information-gathering mission in a nuclear power plant from a Defense Advanced Research Projects Agency (DARPA) robots challenge.
\end{abstract}

\clearpage

\section{Introduction}
Mobile robots---aerial, ground, and aquatic---equipped with sensors and/or cargo have applications in agriculture \cite{santos2020path}, commerce \cite{bogue2016growth}, the delivery of goods \cite{coelho2014thirty}, search-and-rescue \cite{queralta2020collaborative}, chemical, biological, radiological, or nuclear incident response \cite{murphy2012projected}, environmental monitoring \cite{dunbabin2012robots}, industrial chemical plant safety monitoring \cite{soldan2014towards}, forest fire monitoring and fighting \cite{merino2012unmanned}, and military surveillance and reconnaissance. 
Deployment of a \emph{team} of robots can increase spatial coverage and decrease the time to achieve the mission. 
In such cases, a team of mobile robots may coordinate their paths in the environment to cooperatively achieve some shared objective \cite{parker2007distributed}.

In some applications, the team of mobile robots must locomote in an environment that poses risks of robot failures, due to inherently dangerous terrain, harsh weather, the presence of heat, radiation, or corrosive chemicals, or an adversary with the capability to attack robots. Then, the robots may coordinate their paths in consideration of these risks, so that achievement of the team objective is resilient to robot failures \cite{zhou2021multi}. A \emph{resilient} team of robots can (i) withstand failures/attacks with minimal concession of the objective and/or (ii) adapt to attacks/failures of robots on the team to maximize achievement of the objective. 

Several algorithms have been developed for multiple robots to coordinatively plan their paths in risky environments abstracted as graphs (nodes: locations; edges: paths between locations) \cite{jorgensen2018team,shi2023robust,zhou2022distributed}. 
For example, in the Team Surviving Orienteers (TSO) problem \cite{jorgensen2018team}, each node of the graph offers a reward to the team when visited by a robot, but each edge traversal by a robot poses a risk of failure. The objective in the [offline] TOP is to plan the paths of the robots (between two specified nodes) to maximize the expected team reward under the constraint that each robot survives the mission with a probability above a certain threshold. 
Relatedly, the Foraging Route with the Maximum Expected Utility problem \cite{di2022foraging} considers a single robot foraging in an adversarial environment like in the TOP, but the rewards collected by the robot are not accumulated until/unless the robot returns to the home station safely.
In the [offline] Robust Multiple-path Orienteering Problem \cite{shi2023robust}, nodes offer rewards if visited by a robot and that robot returns from the mission; the paths of the $N$ robots are planned to maximize the team reward under the worst-case attacks of $\alpha<N$ of the robots. The optimal path plans must trade off (i) redundancy in the nodes visited to endow robustness against worst-case attacks and (ii) coverage of many nodes to collect many rewards.



\section{The bi-objective risky team orienteering problem (BO-RTOP)}
In the risky team orienteering problem, a team of mobile robots must coordinatively plan closed trails on a directed graph to harvest rewards offered by its nodes, but each edge-traversal by a robot imposes a risk of failure. 
We wish to find the set of Pareto-optimal trail plans for the robot team that maximize (i) the number of robots that survive the mission and (ii) the expected rewards collected by the team.
Offline.


Our problem setup follows Ref.\cite{jorgensen2018team}, except we consider two objectives and omit the constraints that each robot survives with a certain probability.

\subsection{Spatial abstraction of the environment}
We abstract the environment as a directed graph $G=(\mathcal{V}, \mathcal{E})$. Each node $v\in \mathcal{V}$ represents a location. Each edge $(v, v^\prime) \in \mathcal{E}$ represents the best (e.g., shortest or safest) path (in Euclidean space) for a mobile robot to take from node $v$ to node $v^\prime$.
Note, we do not assume the graph is complete\footnote{I.e., not every pair of distinct nodes $\{v, v^\prime\}$ is joined by two edges $(v, v^\prime)$ and $(v^\prime, v)$. E.g., consider one node as the entrance to a hallway and another node as a small room down the hallway. Visiting the room necessitates entering the hallway.}.

\subsection{Mobile robots following trails in a risky environment}
\paragraph{A team of mobile robots.} 
A homogenous team of $K$ mobile robots begin at the base node $v_b \in \mathcal{V}$ of the graph $G$. 

\paragraph{The robot-team trail plans.}
Each robot plans to execute/follow a closed, directed trail $\rho$ on the graph $G$.  
We refer to the set of closed, directed trails $\{\rho_1, ..., \rho_K\}$ the team of robots plans to take as the \emph{robot-team trail plans} for the mission.
These are only ``plans'' because, during execution of them, each robot may fail/get destroyed at some node or edge along its planned trail. Any robot that fails before its trail is completed cannot visit the remainder of the nodes in its planned trail.

\subparagraph{A directed trail.} A directed trail is a sequence of nodes $\rho=(\rho(0), \rho(1), ..., \rho(\ell))$ (so, $\rho(i) \in \mathcal{V}$) such that 
(i) for each node in the sequence except the last one, an edge is directed from that node to the next node in the sequence, i.e., $(\rho(i-1), \rho(i))\in\mathcal{E}$ for $i \in \{1, ..., \ell\}$
and 
(ii) the edges in the multiset of $\ell =\lvert \rho \rvert$ edges $\{(\rho(i-1), \rho(i))\}_{i=1}^{\ell}$ traversed in the trail are distinct.
That the trail is closed means $\rho(0)=\rho(\ell)$, which, here, $=v_b$.
The nodes in the trail need not be distinct.

\subparagraph{The static/offline setting.} During the mission, the robots cannot communicate with each other to be aware of failures, and we cannot send communications to the robots. The robot-team trail plans are set at the beginning of the mission, then followed by the robots without adaptation or updates to the plans during the mission. 

\paragraph{The risks while walking on the graph.}
Each robot incurs a risk of failure/destruction while following its path $\rho$. 
Specifically, starting in a surviving state at some node $v$, a robot survives the lumped process of (i) traversing edge $(v, v^\prime) \in \mathcal{E}$ and (ii) visiting node $v^\prime$ with probability $\omega((v, v^\prime))$. 
We assume (i) each outcome (success or failure) of an edge traversal followed by a node visit by a robot is an independent event and (ii) survival probabilities are static over the course of the mission. 
Thus, the robot survival probability function $\omega: \mathcal{E} \rightarrow [0, 1]$ characterizes the survival probabilities of the $K$ robots with trail plans $\{\rho_1, ..., \rho_K\}$ on the graph $G$.% and (2) the expected utility of the rewards harvested by the robots along their paths, which we write next. 

\subparagraph{Remark on symmetry.} We do not assume $\omega$ is symmetric, i.e., that $\omega((v, v^\prime)) = \omega((v^\prime, v))$. The traversal from node $v$ to $v^\prime$ may be more dangerous than the traversal from $v^\prime$ to $v$ owing to e.g., strong air or water currents in the direction $v^\prime$ to $v$ or a robot sensor at node $v$ that alerts the adversary to attempt to intercept the robot during its outgoing journey. Even if edge traversal risks are symmetric, the action of visiting a node could impose a risk, and node $v$ may be more or less dangerous than node $v^\prime$, breaking symmetry. 

\paragraph{Survival probabilities.} We now write some useful robot survival probabilities based on this risky-traversal model.
\subparagraph{Survival of an individual robot along its trail.}
Let $S_n(\rho) : \{\text{fail}, \text{survive}\} \rightarrow \{0, 1\} $ be the Bernoulli random variable that is $1$ if a robot following trail $\rho$ survives to visit the $n$th node in this trail, and $0$ if it does not. For the event of survival, the robot must survive its traversal of \emph{all} of the first $n$ edges in its path, so:
\begin{equation}
	\pi(S_n(\rho) = 1) = \prod_{i=1}^n \omega((\rho(i-1), \rho(i))) \text{ for } n\in \{0, 1, ..., \lvert \rho \rvert\} \label{eq:pi_S_n}
\end{equation} The factorization owes to the independence of the edge-traversal$+$node-visit events.
The complement of the event of survival is failure, so $\pi(S_n(\rho) = 0)=1-\pi(S_n(\rho) = 1)$.

\subparagraph{Node visit probabilities.} 
Let the random variable $Z_v(\rho)$ be the number of times a robot following path $\rho$ visits node $v\in \mathcal{V} \setminus \{v_b\}$. 
If node $v$ is not in the planned trail, of course, $Z_v=0$. If node $v$ is in the planned trail, then $Z_v(\rho)$ is the sum of random variables $S_n(\rho)$ where $n$ satisfies $\rho(n)=v$. I.e.,
\begin{equation}
	Z_v(\rho) = \sum_{n=1}^{\lvert \rho \rvert} S_n(\rho) \mathcal{I}[\rho(n) = v].
\end{equation}
The number of robots on the team, with trail plans $\{\rho_1,...,\rho_K\}$, that visit node $v$ is $\sum_{k=1}^K Z_v(\rho_k)$.
% HUGE MESS: if trails not paths, then can visit a node more than once. think of a star graph. then this is not a sum of independent variables.

The probability that node $v$ is visited by robot following trail $\rho$ is zero if node $v$ is not in the trail $\rho$ and, otherwise, the probability that the robot survives to visit the node the first time it occurs in the trail:
\begin{equation}
	\pi(Z_v(\rho) > 0)= 
	\begin{cases}
	 0, & v \notin \rho \\
	 \pi(S_{\min_{i} \{i : \rho(i) =v\} } (\rho) = 1) & v \in \rho
	\end{cases} \label{eq:node_visisted}
\end{equation}

\subparagraph{Survival of the team of robots along their trails.}
Let the random variable $R(\{\rho_1, ..., \rho_K\})$ be the number of robots that survive the mission when the robot-team trail plans are $\{\rho_1, ..., \rho_K\}$. The image of $R$ is $\text{Im}[R] = \{0, ..., K\}$, and it is the sum of the individual robot survival [Bernoulli] random variables:
\begin{equation}
	R(\{\rho_1, ..., \rho_K\})=\sum_{k=1}^K S_{\lvert \rho_k \rvert}(\rho_k). \label{eq:R_sum}
\end{equation}
Thus, $R$ is distributed according to the Poisson-Binomial distribution \cite{tang2023poisson}:
\begin{equation}
	\pi(R=r) = \sum_{\substack{\mathcal{R} \subseteq \{1, ..., K\} :  \\ \lvert \mathcal{R} \rvert = r} } 
	\prod_{k \in \mathcal{R}} \pi(S_{\lvert \rho_k \rvert}(\rho_k) = 1)
	\prod_{k \in \{1, ..., K\} \setminus \mathcal{R}} [1- \pi(S_{\lvert \rho_k \rvert}(\rho_k) = 1)], % \text{ for } r \in \{0, ..., K\}
	\label{eq:R_pb}
\end{equation} with $ \pi(S_{\lvert \rho_k \rvert}(\rho_k) = 1)$ in eqn.~\ref{eq:pi_S_n}.
For a given number of robots that survive $r \in \{0, ..., K\}$, eqn.~\ref{eq:R_pb} sums over all possible $\binom{K}{r}$ combinations of the robots that could have survived the mission; the first product is the probability that all of those robots would survive their trails, and the second product is the probability that the other robots fail somewhere along their trails.
Seen from eqn.~\ref{eq:R_sum}, the expectation of $R$ is:
\begin{equation}
	\mathbb{E}[R(\{\rho_1, ..., \rho_K\})]=\sum_{i=1}^K \mathbb{E}[S_{\lvert \rho_i \rvert}(\rho_i)] = \sum_{i=1}^K  \pi(S_{\lvert \rho_k \rvert}(\rho_k) = 1).
\end{equation}

\subsection{Reward structure}
Finally, the reason the robots are following their trails on the graph is to collect rewards from the nodes.
Each node $v\in \mathcal{V}$ in the graph $G$ offers rewards to the team if visited by robot(s) on the team. 
Let $u_v: \{0, ..., K\} \rightarrow \mathbb{R}_{\geq 0}$ be a utility function that maps the number of robots that visit node $v$ over the course of the mission to the reward accumulated by the team. 
Note, the reward is immediately and irrevocably accumulated by the team after a node-visit by a robot on the team---even if the robot fails after leaving this node.
Then, the team-rewards $U(\{\rho_1,...,\rho_K\})$ collected over the course of the mission by the robot-team following trail plans $\{\rho_1, ..., \rho_K\}$ is the random variable:
\begin{equation}
U(\{\rho_1,...,\rho_K\}) = \sum_{v\in\mathcal{V}\setminus \{v_b\}} u_v\left ( \sum_{k=1}^K Z_v(\rho_k) \right).
\end{equation}
(The base node $v_b$ does not offer reward to the team.)
%Generally, the expectation of $U$ is:
%\begin{equation}
%	\mathbb{E}[U(\{\rho_1,...,\rho_K\})]=
%\end{equation}

\subparagraph{Single-visit rewards.}
In the case of single-visit rewards, a node can offer reward to only a single robot. I.e., once a node is visited, it does not offer further rewards to the team. Then, the utility function is
\begin{equation}
	u_v(k) = \begin{cases}
		0 & k = 0 \\
		\upsilon_v & k \geq 1,
	\end{cases}
\end{equation} where $\upsilon_v \in \mathbb{R}_{\geq 0}$ is the reward offered by node $v$. And, the expected team-reward is:
\begin{equation}
	\mathbb{E}[U(\{\rho_1, ..., \rho_K\}) = \sum_{v \in \mathcal{V} \setminus \{v_b\}} \upsilon_v \left(1 - \prod_{k=1}^K [1 - \pi(Z_v(\rho_k) >0)] \right)
\end{equation} where $\pi(Z_v(\rho) >0)$ in eqn.~\ref{eq:node_visisted} gives the probability node $v$ is visited by a robot following trail $\rho$. 
The quantity $1 - \pi(Z_v(\rho_k) >0)$ is the probability that robot $k$ does not visit node $v$. The product is then the probability that none of the robots visit node $v$. If not none of the robots visited node $v$, then some robot visited it; hence, the quantity $(\cdots)$ is the probability that some robot visits node $v$ during the mission.

\subparagraph{Multi-visit rewards.}

\subsection{The bi-objective optimization problem}
We wish to find the team-robot trail plans $\{\rho_1, ..., \rho_K\}$ that maximize two objectives, (1) the expected team-reward and (2) the expected number of robots that survive the mission:
\begin{equation}
\max_{\{\rho_1, ..., \rho_K\}} [\mathbb{E}[U(\{\rho_1, ..., \rho_K\})], \mathbb{E}[R(\{\rho_1, ..., \rho_K\})]].
\end{equation}
These two objectives may compete, in the sense that the ultimate train-plans chosen for the mission may depend on the tradeoff between the expected team-reward and the expected survivability. Consequently, there may not exist a utopian team trails plan that simultaneously maximizes both objectives, and instead we seek the Pareto-optimal set of team-robot trail plans. Then, we present the Pareto set of solutions to the decision-maker, who ultimately chooses the plans according to his or her values in terms of the importance of team-reward vs. robot survival. 

\paragraph{Pareto-optimal team-robot trail plans.} A team trail plan $\{\rho_1, ..., \rho_K\}$ Pareto-dominates team trail plan  $\{\rho_1^\prime, ..., \rho_K^\prime\}$ if both
\begin{align}
	\mathbb{E}[U(\{\rho_1, ..., \rho_K\})] & \geq \mathbb{E}[U(\{\rho_1^\prime, ..., \rho_K^\prime\})]  \text{ and} \\
	\mathbb{E}[R(\{\rho_1, ..., \rho_K\})] & \geq \mathbb{E}[R(\{\rho_1^\prime, ..., \rho_K^\prime\})] 
\end{align}
and either
\begin{align}
	\mathbb{E}[U(\{\rho_1, ..., \rho_K\})] &> \mathbb{E}[U(\{\rho_1^\prime, ..., \rho_K^\prime\})]  \text{ or} \\
	\mathbb{E}[R(\{\rho_1, ..., \rho_K\})] & > \mathbb{E}[R(\{\rho_1^\prime, ..., \rho_K^\prime\})] .
\end{align}
A team trail plan $\{\rho_1, ..., \rho_K\}$ is Pareto-optimal if no other team trail plan Pareto-dominates it.


% harvest risk different from visit risk. once harvested, then no longer risk for other robots to visit that node.

\section{Bi-objective ant colony optimization}
stigmergy

\paragraph{Heuristic}

\paragraph{Pheremone}

\paragraph{Constructing partial solutions}

\paragraph{Laying pheremone}

\section{Results}

\paragraph{DARPA power plant}

Defense Advanced Research Projects Agency (DARPA) Subterranean (SubT) Challenge \cite{chung2023into}

\paragraph{Solutions with different numbers of robots}

\paragraph{Adding adversary with guard}

\section{Discussion}

Fuel constraints.

other risk metrics \cite{majumdar2020should}.

Future work. online


\bibliography{refs}
\bibliographystyle{unsrt}

\end{document}  
