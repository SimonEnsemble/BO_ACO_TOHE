\documentclass[11pt, oneside]{article}  
% \documentclass[fleqn,10pt]{wlpeerj}


\usepackage{geometry}  
%\usepackage{nunito}
\usepackage{cmbright}
\geometry{letterpaper}   
%\usepackage{cite}
\usepackage{graphicx}				% Use pdf, png, jpg, or eps§ with pdflatex; use eps in DVI mode
								% TeX will automatically convert eps --> pdf in pdflatex	
\usepackage{tcolorbox}
\usepackage{amssymb}
\usepackage{longtable}
\usepackage{amsmath}
\usepackage{booktabs}
% \usepackage{emoji}
\usepackage{url}
\usepackage{color}
\definecolor{c1}{rgb}{0.12, 0.56, 1.0}
%SetFonts
\usepackage{xspace}
\usepackage{xcolor}
\usepackage{caption}
\usepackage{subcaption}
\usepackage{authblk} 

\usepackage{sectsty} 
\definecolor{cool_blue}{RGB}{24, 132, 193}
\sectionfont{\color{c1}\selectfont}
\subsectionfont{\color{c1}\selectfont}
\subsubsectionfont{\color{c1}\selectfont}
% \paragraphfont{\color{gray}\selectfont}
\subparagraphfont{\color{gray}\selectfont}

\definecolor{fruitpushorange}{RGB}{255, 127, 0}
\newcommand{\data}{$(s_1, ..., s_k)$\xspace}

\usepackage{soul}
\DeclareRobustCommand{\cms}[1]{ {\begingroup\sethlcolor{fruitpushorange}\hl{(cms:) #1}\endgroup} }
%SetFonts


\title{Bi-objective ant colony optimization of the risky, robot-team orienteering problem}
\author[1]{Cory M. Simon}
\author[2]{Jeffrey Richley}
\author[2]{Lucas Overbey}
\author[2]{Darleen Perez-Lavin}
\affil[1]{School of Chemical, Biological, and Environmental Engineering. Oregon State University. Corvallis, OR. USA.}
% \affil[]{\texttt{cory.simon@oregonstate.edu}}
\affil[2]{Naval Information Warfare Center Atlantic. Charleston, SC. USA.}
% \corrauthor[1]{Cory M. Simon}{cory.simon@oregonstate.edu}

% \keywords{Keyword1, Keyword2, Keyword3}



%\flushbottom
% \maketitle
%\thispagestyle{empty}


%\affil[*]{}
% \date{}							% Activate to display a given date or no date

\begin{document}
\maketitle

\begin{abstract}
In many applications e.g., delivery, patrolling, and information-gathering, a team of mobile [aerial, ground, or aquatic] robots must coordinate their trails in some environment to cooperatively achieve some team-level objective. 
Robots may risk failure/destruction while traversing some environments, owing to dangerous conditions or the presence of adversaries capable of attacking them. 
Then, robot trail-planning should account for these risks.

Herein, we use ant colony optimization to find the [approximate] Pareto-optimal set of robot trail plans for the bi-objective, risky team orienteering problem, where (i) a team of robots are mobile within an environment abstracted as a graph (nodes: locations, edges: paths between locations); (ii) each node offers a reward to the team when visited by a robot; (iii) the traversal of each edge imposes a risk of robot failure/destruction; and (iv) the two [often, competing] team objectives are to maximize the expected (a) team reward and (b) number of robots that survive the mission.
Presenting the Pareto-optimal set of robot trail plans to the downstream decision-maker allows them to choose the plans that balance, according to their values, the two objectives. 
As a case study, we illustrate with an information-gathering mission in a nuclear power plant from a Defense Advanced Research Projects Agency (DARPA) robots challenge.
\end{abstract}

\clearpage

\section{Introduction}
Mobile robots---aerial, ground, and aquatic---equipped with sensors and/or cargo have applications in agriculture \cite{santos2020path}, commerce \cite{bogue2016growth}, the delivery of goods \cite{coelho2014thirty}, search-and-rescue \cite{queralta2020collaborative}, chemical, biological, radiological, or nuclear incident response \cite{murphy2012projected}, environmental monitoring \cite{dunbabin2012robots}, industrial chemical plant safety monitoring \cite{soldan2014towards}, forest fire monitoring and fighting \cite{merino2012unmanned}, and military surveillance and reconnaissance. 
Deployment of a \emph{team} of robots can increase spatial coverage and decrease the time to achieve the mission. 
In such cases, a team of mobile robots may coordinate their paths in the environment to cooperatively achieve some shared objective \cite{parker2007distributed}.

In some applications, the team of mobile robots must locomote in an environment that poses risks of robot failures, due to inherently dangerous terrain, harsh weather, the presence of heat, radiation, or corrosive chemicals, or an adversary with the capability to attack robots. Then, the robots may coordinate their paths in consideration of these risks, so that achievement of the team objective is resilient to robot failures \cite{zhou2021multi}. A \emph{resilient} team of robots can (i) withstand failures/attacks with minimal concession of the objective and/or (ii) adapt to attacks/failures of robots on the team to maximize achievement of the objective. 

Several algorithms have been developed for multiple robots to coordinatively plan their paths in risky environments abstracted as graphs (nodes: locations; edges: paths between locations) \cite{jorgensen2018team,shi2023robust,zhou2022distributed}. 
For example, in the Team Surviving Orienteers (TSO) problem \cite{jorgensen2018team}, each node of the graph offers a reward to the team when visited by a robot, but each edge traversal by a robot poses a risk of failure. The objective in the [offline] TOP is to plan the paths of the robots (between two specified nodes) to maximize the expected team reward under the constraint that each robot survives the mission with a probability above a certain threshold. 
Relatedly, the Foraging Route with the Maximum Expected Utility problem \cite{di2022foraging} considers a single robot foraging in an adversarial environment like in the TOP, but the rewards collected by the robot are not accumulated until/unless the robot returns to the home station safely.
In the [offline] Robust Multiple-path Orienteering Problem \cite{shi2023robust}, nodes offer rewards if visited by a robot and that robot returns from the mission; the paths of the $N$ robots are planned to maximize the team reward under the worst-case attacks of $\alpha<N$ of the robots. The optimal path plans must trade off (i) redundancy in the nodes visited to endow robustness against worst-case attacks and (ii) coverage of many nodes to collect many rewards.



\section{The bi-objective risky team orienteering problem (BO-RTOP)}
In the risky team orienteering problem, a team of mobile robots must coordinatively plan closed trails on a directed graph to harvest rewards offered by its nodes, but each edge-traversal by a robot imposes a risk of failure. 
We wish to find the set of Pareto-optimal trail plans for the robot team that maximize (i) the number of robots that survive the mission and (ii) the expected rewards collected by the team.
Offline.

Rewards 

\subsection{Risky team orienteering}

\paragraph{Spatial abstraction of the environment.}
We abstract the environment as a directed graph $G=(\mathcal{V}, \mathcal{E})$. Each node $v\in \mathcal{V}$ represents a location. Each edge $(v, v^\prime) \in \mathcal{E}$ represents the best (e.g., shortest or safest) path (in Euclidean space) for a mobile robot to take from node $v$ to node $v^\prime$.
Note, we do not assume the graph is complete\footnote{I.e., not every pair of distinct nodes $\{v, v^\prime\}$ is joined by two edges $(v, v^\prime)$ and $(v^\prime, v)$. E.g., consider one node as the entrance to a hallway and another node as a small room down the hallway. Visiting the room necessitates entering the hallway.}.

\paragraph{A team of mobile robots.} 
A homogenous team of $K$ mobile robots begin at the base node $v_b \in \mathcal{V}$ of the graph $G$. 

\paragraph{The robot-team trail plans.}
Each robot plans to execute/follow a closed, directed trail $\rho$ on the graph $G$.  
We refer to the set of closed, directed trails $\{\rho_1, ..., \rho_K\}$ the team of robots plans to take as the \emph{robot-team trail plans} for the mission.
These are only ``plans'' because, during execution of them, each robot may fail/get destroyed at some node or edge along its planned trail. Any robot that fails before its trail is completed cannot visit the remainder of the nodes in its planned trail.

\subparagraph{A directed trail.} A directed trail is a sequence of nodes $\rho=(\rho(0), \rho(1), ..., \rho(\ell))$ such that 
(i) for each node in the sequence except the last, an edge is directed from that node to the next node in the sequence, i.e., $(\rho(i-1), \rho(i))\in\mathcal{E}$ for $i \in \{1, ..., \ell\}$
and 
(ii) the edges in the multiset of $\ell =\lvert \rho \rvert$ edges $\{(\rho(i-1), \rho(i))\}_{i=1}^{\ell}$ traversed in the trail are distinct.
That the trail is closed means $\rho(0)=\rho(\ell)$, which, here, $=v_b$.
The nodes in the trail need not be distinct.

\subparagraph{The static/offline setting.} During the mission, the robots cannot communicate with each other to be aware of failures, and we cannot send communications to the robots. The robot-team trail plans are set at the beginning of the mission, then followed by the robots without adaptation or updates to the plans during the mission. 

\paragraph{Survival probabilities.}
Each robot incurs a risk of failure/destruction while following its path $\rho$. 
Specifically, starting in a surviving state at some node $v$, a robot survives the lumped process of (i) traversing edge $(v, v^\prime) \in \mathcal{E}$ and (ii) visiting node $v^\prime$ with probability $\omega((v, v^\prime))$. 
We assume (i) each outcome (success or failure) of an edge traversal followed by a node visit by a robot is an independent event and (ii) survival probabilities are static over the course of the mission. 
Thus, the robot survival probability function $\omega: \mathcal{E} \rightarrow [0, 1]$ characterizes the survival probabilities of the $K$ robots with trail plans $\{\rho_1, ..., \rho_K\}$ on the graph $G$.% and (2) the expected utility of the rewards harvested by the robots along their paths, which we write next. 

\subparagraph{Remark on symmetry.} Note, we do not assume $\omega$ is symmetric, i.e., that $\omega((v, v^\prime)) = \omega((v^\prime, v))$. The traversal from node $v$ to $v^\prime$ may be more dangerous than the traversal from $v^\prime$ to $v$ owing to e.g., strong air or water currents in the direction $v^\prime$ to $v$, a robot sensor at node $v$ that alerts the adversary to attempt to intercept the robot in its outgoing journey. More, even if edge traversal risks are symmetric, the action of visiting a node could impose a risk, and node $v$ may be more or less dangerous than node $v^\prime$, breaking symmetry. 

\subparagraph{An individual robot.}
Let $S_n(\rho) : \{\text{fail}, \text{survive}\} \rightarrow \{0, 1\} $ be the Bernoulli random variable that is $1$ if a robot following trail $\rho$ survives to visit node $n \in \{0, 1, ..., \lvert \rho \rvert\}$ in the trail $\rho$ and $0$ otherwise. For the survival event, the robot must survive its traversal of \emph{all} of the first $n$ edges in the path, so:
\begin{equation}
	\pi(S_n(\rho) = 1) = \prod_{i=1}^n \omega((\rho(i-1), \rho(i))).
\end{equation} The factorization owes to the independence of the edge-traversal$+$node-visit event
The complement of the event of survival is failure, so $\pi(S_n(\rho) = 0)=1-\pi(S_n(\rho) = 1)$.

\subparagraph{The team of robots.}
Let the random variable $R(\{\rho_1, ..., \rho_K\})$ be the number of robots that survive the mission when the robot-team trail plans are $\{\rho_1, ..., \rho_K\}$. The image of $R$ is $\text{Im}[R] = \{0, ..., K\}$, and it is the sum of the individual robot survival random variables:
\begin{equation}
	R(\{\rho_1, ..., \rho_K\})=\sum_{i=1}^K S_{\lvert \rho_i \rvert}(\rho_i).
\end{equation}
Thus, the expectation of $R$ is:
\begin{equation}
	\mathbb{E}[R(\{\rho_1, ..., \rho_K\})]=\sum_{i=1}^K \mathbb{E}[S_{\lvert \rho_i \rvert}(\rho_i)] = \sum_{i=1}^K  \prod_{i=1}^{\lvert \rho_i\rvert } \omega((\rho(i-1), \rho(i))).
\end{equation}
Note, $R$ follows a Poisson Binomial distribution, but we will only use the probability that $R=0$, i.e. that no robots survive. 

\subparagraph{Node-visit probabilities.}
We are interested in the Bernoulli random variable $T_v(\{ \rho_1, ..., \rho_K \} ):\{ \text{not visisted}, \text{visited} \} \rightarrow \{0, 1\}$, which is 1 if node $v$ is visited by \emph{any} robot (one or more) on the team following trail-plans $\{ \rho_1, ..., \rho_K \}$ and 0 otherwise (i.e., if not visited by any robot).
First, let the random variable $\Theta_v(\{ \rho_1, ..., \rho_K \} )$ be the number of robots on the team following trail-plans $\{ \rho_1, ..., \rho_K \}$ that visit node $v$:
\begin{equation}
	\Theta_v(\{ \rho_1, ..., \rho_K \} )= \sum_{i=1}^K \sum_{n=1}^{\lvert \rho_i \rvert} S_n(\rho_i) \mathcal{I}[\rho(n)=v].
\end{equation}
Then the event $T_v=0$ is equivalent to $\Theta_v=0$.


\paragraph{Reward structure.}
Each node $v\in \mathcal{V}$ in the graph $G$ offers a reward to be harvested from it. 
The reward function $r: \mathcal{V} \rightarrow \mathbb{R}_{\geq 0}$ scores each node according to the utility the team of robots gains by harvesting its reward. The utility $r(v)$ is irrevocably accumulated by the team of robots if a robot from the team visits node $v$ during the mission.
Once the reward of a node is harvested by a visiting robot, the node does not offer any further reward to the robot team.

The accumulated reward $A(\{\rho_1, ..., \rho_K\})$ is a random variable dependent on the team-robot trail plans.
The stochasticity in $A$ emanates from the stochastic nature of robot survival when following their paths. 

\begin{equation}
	A(\{\rho_1, ..., \rho_K\}) = \sum_{v=1}^{ \lvert \mathcal{V} \rvert} r(v)%\mathcal{I}[v \in 
\end{equation}


% harvest risk different from visit risk. once harvested, then no longer risk for other robots to visit that node.


\paragraph{Expected number of robots that survive the mission.}

\paragraph{Expected team rewards.}

\subsection{The bi-objective optimization problem}

\section{Bi-objective ant colony optimization}
stigmergy

\paragraph{Heuristic}

\paragraph{Pheremone}

\paragraph{Constructing partial solutions}

\paragraph{Laying pheremone}

\section{Results}

\paragraph{DARPA power plant}

Defense Advanced Research Projects Agency (DARPA) Subterranean (SubT) Challenge \cite{chung2023into}

\paragraph{Solutions with different numbers of robots}

\paragraph{Adding adversary with guard}

\section{Discussion}

Fuel constraints.

other risk metrics \cite{majumdar2020should}.

Future work. online


\bibliography{refs}
\bibliographystyle{unsrt}

\end{document}  
